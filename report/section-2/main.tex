%{{{ Preamble
\documentclass[12pt,a4paper,letterpaper]{article}
%}}}
%{{{ Packages
%{{{ geometry
\usepackage[
    bottom=2cm,
    left=3cm,
    right=2cm,
    top=3cm,
]{geometry}
%}}}
%{{{ bibliography
\usepackage[
    backend=bibtex,
    defernumbers=true,
    encoding=latin1,
    style=abnt,
]{biblatex}
\usepackage[autostyle]{csquotes}
%}}}
%{{{ type input font
\usepackage[T1]{fontenc}
\usepackage[brazil]{babel}
\usepackage[brazil]{varioref}
\usepackage[utf8]{inputenc}
%}}}
%{{{ type output font
\usepackage{
    amsfonts,
    amsmath,
    amsopn,
    amssymb,
    amsthm,
    latexsym
}
\usepackage{indentfirst}
%}}}
%}}}
%{{{ Add bib
\addbibresource{bib/articles.bib}
%}}}
%{{{ Add New Command
\newcommand\wb[1]{\discretionary{#1}{#1}{#1}}
%}}}
%{{{ Documento
\begin{document}

\section{Identificação do problema no fluxo da classificação}
Temos indícios de que podemos melhorar a classificação na escolha de feature
para o pre-processamento da base, visto que com algoritmos genético e SVM,
podemos classificar de forma competitiva, mas sabemos que o algoritmo está
interagindo com os parâmetros do SVM. Também em mão uso de aceleradores ou
impulsores na classificação usando técnica fuzzy e SVM.

Podemos avaliar tais combinações de técnicas e analisar tais resultados, antes
mesmo de testarmos a nossa hipótese, de que, só ajustando os detectores de
características e fixarmos tais parâmetros do SVM melhorará os resultados de
classificação abruptamente.

\pagebreak
%{{{ Referências Bibliográficas
\medskip
\printbibliography[
    heading=bibintoc,
    title={Referências Bibliográficas}
]
%}}}
\end{document}
%}}}
