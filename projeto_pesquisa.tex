\documentclass[12pt,a4paper,oneside]{book}

\usepackage[
    bottom=2cm,
    left=3cm,
    right=2cm,
    top=3cm,
]{geometry}

\usepackage[T1]{fontenc}
\usepackage[active]{srcltx}
\usepackage[all]{xy}
\usepackage[autostyle]{csquotes}
\usepackage[brazil]{babel}
\usepackage[brazil]{varioref}
\usepackage[hidelinks]{hyperref}
\usepackage[latin1]{inputenc}
\usepackage[rflt]{floatflt}
\usepackage{alltt}
\usepackage{amsfonts,amsthm,amsopn,amssymb,latexsym}
\usepackage{amsmath}
\usepackage{appendix}
\usepackage{blindtext}
\usepackage{caption}
\usepackage{colortbl}
\usepackage{dcolumn}
\usepackage{epic}
\usepackage{epsfig,longtable}
\usepackage{epstopdf}
\usepackage{graphicx}
\usepackage{indentfirst}
\usepackage{makeidx}
\usepackage{microtype}
\usepackage{multirow}
\usepackage{pdfpages}
\usepackage{pifont}
\usepackage{ragged2e}
\usepackage{setspace}
\usepackage{subfig}
\usepackage{textcomp}
\usepackage{url}
\usepackage{wrapfig}
\usepackage{xcolor}
\usepackage{xr}

\usepackage[
    backend=bibtex,
    defernumbers=true,
    encoding=latin1,
    style=abnt,
]{biblatex}

\addbibresource{referencias/papers.bib}

\newcommand\wb[1]{\discretionary{#1}{#1}{#1}}


\begin{document}
\documentclass{article}
\usepackage[a4paper]{geometry}

\usepackage{tikz}

\usetikzlibrary{mindmap,trees,shadows,decorations}

\usepackage{verbatim}

\begin{document}

\pagestyle{empty}

\begin{tikzpicture}
  \path[mindmap,
        align=flush center,
        grow cyclic,
        every node/.style=concept,
        concept color=black,
        text=white,
        font=\fontsize{20pt}{20pt}\selectfont]

    node {Pesquisa científica} [clockwise from=0]
    child[concept color=orange!80!black] {
      node(c1) {Classificação da pesquisa científica}
      [clockwise from=90]
      child { node[concept](c11) {Pesquisa Exploratória} }
      child { node[concept](c12) {Pesquisa descritivas} }
      child { node[concept](c13) {Pesquisa explicativas} }
    }
    child[concept color=red!80!black] {
      node[concept] {Problema de pesquisa}(c2)
    }
\end{tikzpicture}


\begin{tikzpicture}
  \path[mindmap,
        align=flush center,
        grow cyclic,
        every node/.style=concept,
        concept color=orange!80!black,
        text=white,
        font=\fontsize{20pt}{20pt}\selectfont]
    node {Pesquisa Exploratória} [clockwise from=0]
    child[concept] { node(c1) {Pesquisa bibliográfica} }
    child[concept] { node(c2) {Pesquisa documental} }
\end{tikzpicture}

\begin{tikzpicture}
  \path[mindmap,
        align=flush center,
        grow cyclic,
        every node/.style=concept,
        concept color=orange!80!black,
        text=white,
        font=\fontsize{20pt}{20pt}\selectfont]

    node {Pesquisa Descritivas} [clockwise from=0]
    child[concept color=orange!80!black] {
      node(c1) {}
    }
\end{tikzpicture}

\begin{tikzpicture}
  \path[mindmap,
        align=flush center,
        grow cyclic,
        every node/.style=concept,
        concept color=orange!80!black,
        text=white,
        font=\fontsize{20pt}{20pt}\selectfont]

    node {Pesquisa explicativas} [clockwise from=0]
    child[concept color=orange!80!black] {
      node(c1) {Pesquisa experimental}
    }
\end{tikzpicture}


\begin{tikzpicture}
  \path[mindmap,
        align=flush center,
        grow cyclic,
        every node/.style=concept,
        concept color=red!80!black,
        text=white,
        font=\fontsize{20pt}{20pt}\selectfont]
    node {Problema de pesquisa} [clockwise from=-120]
    child[concept color=red!80!black] {
      node(c1) {}
      [clockwise from=0]
      child { node[concept](c11) {} }
    }
\end{tikzpicture}

\end{document}

\thispagestyle{empty}

\frontmatter

\tableofcontents
\pagebreak

\addcontentsline{toc}{chapter}{Indentifica\c{c}\~{a}o do Plano}
\chapter*{Identifica\c{c}\~{a}o do Plano}
\renewcommand{\labelenumi}{\alph{enumi}}
\begin{enumerate}
    \item - Discente: \textbf{Iury Adones Xavier dos Santos}
    \item - Bolsista: Sim ( )  N\~{a}o ( \textbf{x} )
    \item - N\'{i}vel: \textbf{Mestrado}
    \item - Entrada (ano/semestre): \textbf{2017/1}
    \item - Orientador: \textbf{Dr. Adenilton Jos\'{e} da Silva}
    \item - Co-orientador: \textbf{Dr. P\'{e}ricles Barbosa Cunha de Miranda}
    \item - Projeto de pesquisa do orientador ao qual est\'{a} vinculado o plano
        (projeto guarda\wb-chuva): \textbf{Recomenda\c{c}\~{a}o de algoritmos
        evolucion\'{a}rios para problemas de otimiza\c{c}\~{a}o com base em
        caracter\'{i}sticas da superf\'{i}cie de fitness}
    \item - \'{A}rea de concentra\c{c}\~{a}o: \textbf{Computa\c{c}\~{a}o
        Inteligente e Modelagem}
    \item - Linha de pesquisa: \textbf{Aprendizagem de M\'{a}quina, Vis\~{a}o
        Computacional, Reconhecimento de Padr\~{o}es.}
\end{enumerate}
\renewcommand{\labelenumi}{\arabic{enumi}}
\pagebreak

\mainmatter

\chapter{Introdu\c{c}\~{a}o}
\textcolor{red}{(Deve constar obrigatoriamente revis\~{a}o de literatura,
vincula\c{c}\~{a}o com a linha de pesquisa escolhida e o problema da pesquisa.
Obs: exemplo da refer\^{e}ncia \autocite{einstein}, \cite[10]{einstein},
\citeauthor{einstein} e \textcite{einstein})}
\pagebreak

\chapter{Objetivos}
\textcolor{red}{(deve conter verbos sempre no infinitivo)}
\section{Geral}
\'{E} investigar se a meta-heur\'{i}stica possui maior viabilidade para
solu\c{c}\~{a}o de um problema de otimiza\c{c}\~{a}o.
\section{Espec\'{i}fico}
Usar t\'{e}cnicas de pr\'{e}\wb-processamento de imagens e aprendizagens de
m\'{a}quinas em imagens de caracteres e d\'{i}gitos. No entanto, tais
t\'{e}cnicas aplicadas tem uma gama de par\^{a}metros a serem otimizados,
sabendo que o objetivo se caracteriza em buscar a melhor extra\c{c}\~{a}o de
caracteristicas para que os algoritmos de classifica\c{c}\~{a}o esteja em seu
melhor estado, entretanto os algoritmos de aprendizagem de m\'{a}quinas
tamb\'{e}m tem par\^{a}metros que poder\'{a} serem otimizados h\'{a} s\'{o}s ou
glutinando diferentes tecnicas de aprendizagem de maquinas, com isso guardaremos
os pesos para cada resposta de m\'{a}quina na sua classifica\c{c}\~{a}o e iremos
fazer combina\c{c}\~{o}es das redes neurais e tendo em m\~{a}os tais respostas
para usarmos as tecnicas, tamb\'{e}m buscaremos otimizar a classifica\c{c}\~{a}o
com tais abordagens.


\pagebreak

\chapter{Metodologia}
\textcolor{red}{(Material e m\'{e}todos, descrever como ser\'{a} o processo de
coleta e an\'{a}lise de dados, bem como teorias que embasam a an\'{a}lise)}
\section

\pagebreak

\chapter{Resultados Esperados}
\begin{center}
    \begin{tabular}{|l|c|}
        \hline
        \multicolumn{1}{|c|}{Categoria} & \multicolumn{1}{|c|}{N\'{u}mero
        esperado} \\ \hline
        Artigo cient\'{i}fico & 1 \\ \hline
        Disserta\c{c}\~{a}o de Mestrado & 1 \\ \hline
        Livro ou cap\'{i}tulo de livro & 0 \\ \hline
        % Outras (especificar) & \\ \hline
        Participa\c{c}\~{a}o em Editais & 0 \\ \hline
        Participa\c{c}\~{a}o em congresso & 2 \\ \hline
        Patente ou equivalente & 0 \\ \hline
        Tese de Doutorado & 0 \\ \hline
    \end{tabular}
\end{center}
\pagebreak

\chapter{Cronograma do discente}
\begin{center}
    \begin{tabular}{|l|c|c|c|c|c|c|c|c|c|}
        \hline
        \multicolumn{1}{|c|}{\multirow{2}{*}{Atividades}} &
        \multicolumn{8}{c|}{Semestres} \\ \cline{2-9}
         & 1$º$ & 2$º$ & 3$º$ & 4$º$ & 5$º$ & 6$º$ & 7$º$ & 8$º$ \\ \hline
        Cr\'{e}ditos de disciplinas & x & x & & & & & & \\ \hline
        Cr\'{e}ditos de orienta\c{c}\~ao & & x & x & & & & & \\ \hline
        Pesquisa Bibliogr\'{a}fica & & & x & & & & & \\ \hline
        Obten\c{c}\~ao de dados & & & x & &  & & & \\ \hline
        An\'{a}lise de dados & & & x & & & & & \\ \hline
        % Profici\^{e}ncias (ingl\^{e}s/espanhol) & & & & & & & & \\ \hline
        Revis\~{a}o e reda\c{c}\~ao da tese & & & & x & & & & \\ \hline
        Defesa da tese & & & & x & & & & \\ \hline
    \end{tabular}
\end{center}
\pagebreak

\medskip
\printbibliography[heading=bibintoc,title={Refer\^{e}ncias},type=article]

\backmatter
\end{document}
