%{{{ Preamble
\documentclass[12pt,a4paper,oneside]{book}
%}}}
%{{{ Packages
\usepackage[
    bottom=2cm,
    left=3cm,
    right=2cm,
    top=3cm,
]{geometry}

\usepackage[T1]{fontenc}
\usepackage[active]{srcltx}
\usepackage[all]{xy}
\usepackage[autostyle]{csquotes}
\usepackage[brazil]{babel}
\usepackage[brazil]{varioref}
\usepackage[hidelinks]{hyperref}
\usepackage[latin1]{inputenc}
\usepackage[rflt]{floatflt}
\usepackage{alltt}
\usepackage{amsfonts,amsthm,amsopn,amssymb,latexsym}
\usepackage{amsmath}
\usepackage{appendix}
\usepackage{blindtext}
\usepackage{caption}
\usepackage{colortbl}
\usepackage{dcolumn}
\usepackage{epic}
\usepackage{epsfig,longtable}
\usepackage{epstopdf}
\usepackage{graphicx}
\usepackage{indentfirst}
\usepackage{makeidx}
\usepackage{microtype}
\usepackage{multirow}
\usepackage{pdfpages}
\usepackage{pifont}
\usepackage{ragged2e}
\usepackage{setspace}
\usepackage{subfig}
\usepackage{textcomp}
\usepackage{url}
\usepackage{wrapfig}
\usepackage{xcolor}
\usepackage{xr}

\usepackage[
    backend=bibtex,
    defernumbers=true,
    encoding=latin1,
    style=abnt,
]{biblatex}
%}}}
%{{{ Add bib
\addbibresource{referencias/papers.bib}
%}}}
%{{{ Add New Command
\newcommand\wb[1]{\discretionary{#1}{#1}{#1}}
%}}}
%{{{ Documento
\begin{document}
%{{{ capa
\documentclass{article}
\usepackage[a4paper]{geometry}

\usepackage{tikz}

\usetikzlibrary{mindmap,trees,shadows,decorations}

\usepackage{verbatim}

\begin{document}

\pagestyle{empty}

\begin{tikzpicture}
  \path[mindmap,
        align=flush center,
        grow cyclic,
        every node/.style=concept,
        concept color=black,
        text=white,
        font=\fontsize{20pt}{20pt}\selectfont]

    node {Pesquisa científica} [clockwise from=0]
    child[concept color=orange!80!black] {
      node(c1) {Classificação da pesquisa científica}
      [clockwise from=90]
      child { node[concept](c11) {Pesquisa Exploratória} }
      child { node[concept](c12) {Pesquisa descritivas} }
      child { node[concept](c13) {Pesquisa explicativas} }
    }
    child[concept color=red!80!black] {
      node[concept] {Problema de pesquisa}(c2)
    }
\end{tikzpicture}


\begin{tikzpicture}
  \path[mindmap,
        align=flush center,
        grow cyclic,
        every node/.style=concept,
        concept color=orange!80!black,
        text=white,
        font=\fontsize{20pt}{20pt}\selectfont]
    node {Pesquisa Exploratória} [clockwise from=0]
    child[concept] { node(c1) {Pesquisa bibliográfica} }
    child[concept] { node(c2) {Pesquisa documental} }
\end{tikzpicture}

\begin{tikzpicture}
  \path[mindmap,
        align=flush center,
        grow cyclic,
        every node/.style=concept,
        concept color=orange!80!black,
        text=white,
        font=\fontsize{20pt}{20pt}\selectfont]

    node {Pesquisa Descritivas} [clockwise from=0]
    child[concept color=orange!80!black] {
      node(c1) {}
    }
\end{tikzpicture}

\begin{tikzpicture}
  \path[mindmap,
        align=flush center,
        grow cyclic,
        every node/.style=concept,
        concept color=orange!80!black,
        text=white,
        font=\fontsize{20pt}{20pt}\selectfont]

    node {Pesquisa explicativas} [clockwise from=0]
    child[concept color=orange!80!black] {
      node(c1) {Pesquisa experimental}
    }
\end{tikzpicture}


\begin{tikzpicture}
  \path[mindmap,
        align=flush center,
        grow cyclic,
        every node/.style=concept,
        concept color=red!80!black,
        text=white,
        font=\fontsize{20pt}{20pt}\selectfont]
    node {Problema de pesquisa} [clockwise from=-120]
    child[concept color=red!80!black] {
      node(c1) {}
      [clockwise from=0]
      child { node[concept](c11) {} }
    }
\end{tikzpicture}

\end{document}

\thispagestyle{empty}
%}}}
%{{{ Iniciar com Números Romanos
\frontmatter
%}}}
%{{{ Sumário
\tableofcontents
\pagebreak
%}}}
%{{{ Identificação do plano
\addcontentsline{toc}{chapter}{Indentifica\c{c}\~{a}o do Plano}
\chapter*{Identifica\c{c}\~{a}o do Plano}
\renewcommand{\labelenumi}{\alph{enumi}}
\begin{enumerate}
    \item - Discente: \textbf{Iury Adones Xavier dos Santos}
    \item - Bolsista: Sim ( )  N\~{a}o ( \textbf{x} )
    \item - N\'{i}vel: \textbf{Mestrado}
    \item - Entrada (ano/semestre): \textbf{2017/1}
    \item - Orientador: \textbf{Dr. Adenilton Jos\'{e} da Silva}
    \item - Co-orientador: \textbf{Dr. P\'{e}ricles Barbosa Cunha de Miranda}
    \item - Projeto de pesquisa do orientador ao qual est\'{a} vinculado o plano
        (projeto guarda\wb-chuva): \textbf{Recomenda\c{c}\~{a}o de algoritmos
        evolucion\'{a}rios para problemas de otimiza\c{c}\~{a}o com base em
        caracter\'{i}sticas da superf\'{i}cie de fitness}
    \item - \'{A}rea de concentra\c{c}\~{a}o: \textbf{Computa\c{c}\~{a}o
        Inteligente e Modelagem}
    \item - Linha de pesquisa: \textbf{Aprendizagem de M\'{a}quina, Vis\~{a}o
        Computacional, Reconhecimento de Padr\~{o}es.}
\end{enumerate}
\renewcommand{\labelenumi}{\arabic{enumi}}
\pagebreak
%}}}
%{{{ Iniciar com Números Arábico
\mainmatter
%}}}
%{{{ Introdução
\chapter{Introdu\c{c}\~{a}o}
Neste per\'{i}odo a comunidade acad\^{e}mica v\^{e}m discutindo e produzindo
trabalhos relacionados a redes neurais convolucionais profundas, tais como os de
\autocite{JADERBERG2014} e \autocite{ANIL2015}, onde as mesmas det\'{e}m
resultados pr\'{o}ximo ao dos seres humanos em rela\c{c}\~ao a
classifica\c{c}\~ao de objetos nas imagens.

As redes neurais convolucionais profundas aplicadas a imagens, necessita passar
por uma etapa de gera\c{c}\~{a}o de n\'{u}cleos extrator de caracteristicas em
seu processo visto em \autocite{GOODFELLOW2013}, sendo pivor nos resultados de
classifica\c{c}\~{a}o.

Ser\'{a} oportuno usarmos t\'{e}cnicas da Meta\wb-heur\'{i}stica de
acordo com \autocite{MOHAMMAD2017} para otimiza\c{c}\~{a}o na aprendizagem
profunda, nesta etapa que poderemos obter caracter\'{i}sticas de letras,
n\'{u}meros ou objetos contidos numa imagem?

Logo vemos uma oportunidade em contribuir com a comunidade, nesta
etapa de pr\'{e}-processamento e extra\c{c}\~{a}o de caracter\'{i}sticas, o qual
iremos explorar os m\'{e}todos separados, tais como reproduzidos em
\autocite{GUPTA2007}, e implementaremos o m\'{e}todo de exames de part\'{i}culas
adaptativos, visto em \autocite{ZHAN2009}.  Obtendo os resultados em imagens com
caracteres, digitos, e escrita de n\'{u}meros, de bases de pesuisas em vis\~{a}o
computacional \autocite{MNIST2010} e \autocite{CAMPOS2009}.
\pagebreak
%}}}
%{{{ Objetivos
\chapter{Objetivos}
%{{{ Objetivo Geral
\section{Geral}
\'{E} investigar se a meta-heur\'{i}stica possui maior viabilidade para
solu\c{c}\~{a}o de um problema de otimiza\c{c}\~{a}o.
%}}}
%{{{ Objetivo Específico
\section{Espec\'{i}fico}
Reproduzir t\'{e}cnicas de pr\'{e}\wb-processamento e binariza\c{c}\~{a}o de
imagens, utilizar algoritmos de aprendizagens de m\'{a}quinas nos vetores de
caracteristicas das imagens de caracteres e d\'{i}gitos, no entanto
j\'{a} processadas e binarizadas. Tais t\'{e}cnicas aplicadas
t\^{e}m uma gama de par\^{a}metros a serem ajustados, sabendo que o objetivo se
caracteriza em buscar a melhor extra\c{c}\~{a}o de caracteristicas para que os
algoritmos de classifica\c{c}\~{o}es estejam em seus melhores estados.
%}}}
\pagebreak
%}}}
%{{{ Metodologia
\chapter{Metodologia}
Neste cap\'{i}tulo veremos os materiais que ser\~{a}o utilizados na pesquisa e a
descri\c{c}\~{a}o da pesquisa em seu desenvolvimento acad\^{e}mico.
%{{{ Materiais
\section{Materiais}
\noindent
Ser\'{a} usado no desenvolvimento da pesquisa um notebook com as seguintes configura\c{c}\~{o}es:

\begin{itemize}
    \item Processador: Intel Core i5 2410M 2.3/2.9 GHz, 3 MB de cache;
    \item Chipset: Intel HM65;
    \item Placa de v\'{i}deo: Intel HD3000/ Geforce GT520M 1 GB GDDR3;
    \item Mem\'{o}ria ram: 6 GB DDR3 1333 MHz;
    \item Disco r\'{i}gido (SSD): 480 GB SATA Rev. 2.0 (3 GB/s);
    \item Conex\~{o}es: 4 USB 2.0, 1 HDMI, 1 RJ45, 1 VGA, Bluetooth 3.0, fone de
        ouvido, microfone e leitor de cart\~{o}es de mem\'{o}ria 4 em 1.
    \item Rede ethernet: 10/100/1000;
    \item Rede ethernet sem fio: 802.11 b/g/n;
    \item Drive \'{O}ptico: leitor e gravador de CD/DVD;
    \item Tela: 14$"$ LCD LED com resolu\c{c}\~{a}o HD (1366 x 768);
    \item Sistema operacional: Archlinux 64bits;
    \item Bateria: 6 c\'{e}lulas.
\end{itemize}

\noindent
Utlizaremos fontes de imagens p\'{u}plicas de caracteres e n\'{u}meros tais
como:

\begin{itemize}
    \item MNIST \autocite{MNIST2010}
    \item Chars74K \autocite{CAMPOS2009}
\end{itemize}
%}}}
%{{{ Descrição da Pesquisa
\section{Desenvolvimento da Pesquisa}
Implementaremos m\'{e}todos de pr\'{e}\wb-processamento e binariza\c{c}\~{a}o de
imagens tais estabelecidos em \autocite{GUPTA2007} nas imagens coletadas, com
isso passaremos a modificar os par\^{a}metros com o met\'{o}do de enxames de
part\'{i}culas adaptativo reproduzido por \autocite{ZHAN2009}. O resultado das
combina\c{c}\~{o}es nos dar imagens processadas, no entanto guardaremos as
imagens resultantes, pois utilizaremos com outros algoritmos de
pr\'{e}\wb-processamento, sendo assim teremos mais combina\c{c}\~{o}es do
espa\c{c}o de busca de acordo com \autocite{KATKAR2015}.

Utilizaremos algoritmos descritores de imagens reproduzidos por
\autocite{SAWANT2016} e \autocite{NEWELL2011}, obtendo as respostas dos descritores,
ir\'{a} formar a nossa base de caracter\'{i}sticas, no entanto teremos 1000
exemplos de cada imagem pr\'{e}\wb-processada e binarizada, no qual obteremos as
caracteriscas, logo consiguiremos utilizar m\'{e}todos de Redes neurais tais
como \autocite{KATKAR2015} e usando as m\'{e}tricas de avalia\c{c}\~{o}es dos
mesmos.
%}}}
\pagebreak
%}}}
%{{{ Resultados Esperados
\chapter{Resultados Esperados}
\begin{center}
    \begin{tabular}{|l|c|}
        \hline
        \multicolumn{1}{|c|}{Categoria} & \multicolumn{1}{|c|}{N\'{u}mero
        esperado} \\ \hline
        Artigo cient\'{i}fico & 1 \\ \hline
        Disserta\c{c}\~{a}o de Mestrado & 1 \\ \hline
        Livro ou cap\'{i}tulo de livro & 0 \\ \hline
        % Outras (especificar) & \\ \hline
        Participa\c{c}\~{a}o em Editais & 0 \\ \hline
        Participa\c{c}\~{a}o em congresso & 2 \\ \hline
        Patente ou equivalente & 0 \\ \hline
        Tese de Doutorado & 0 \\ \hline
    \end{tabular}
\end{center}
\pagebreak
%}}}
%{{{ Cronogroma do discente
\chapter{Cronograma do discente}
\begin{center}
    \begin{tabular}{|l|c|c|c|c|c|c|c|c|c|}
        \hline
        \multicolumn{1}{|c|}{\multirow{2}{*}{Atividades}} &
        \multicolumn{8}{c|}{Semestres} \\ \cline{2-9}
         & 1$º$ & 2$º$ & 3$º$ & 4$º$ & 5$º$ & 6$º$ & 7$º$ & 8$º$ \\ \hline
        Cr\'{e}ditos de disciplinas & x & x & & & & & & \\ \hline
        Cr\'{e}ditos de orienta\c{c}\~ao & & x & x & & & & & \\ \hline
        Pesquisa Bibliogr\'{a}fica & & & x & & & & & \\ \hline
        Obten\c{c}\~ao de dados & & & x & &  & & & \\ \hline
        An\'{a}lise de dados & & & x & & & & & \\ \hline
        % Profici\^{e}ncias (ingl\^{e}s/espanhol) & & & & & & & & \\ \hline
        Revis\~{a}o e reda\c{c}\~ao da tese & & & & x & & & & \\ \hline
        Defesa da tese & & & & x & & & & \\ \hline
    \end{tabular}
\end{center}
\pagebreak
%}}}
%{{{ Referências Bibliográficas
\medskip
\printbibliography[
    heading=bibnumbered,
    title={Refer\^{e}ncias},
]
% \printbibliography[
%     heading=subbibnumbered,
%     title={Artigos},
%     type=article,
% ]
% \printbibliography[
%     heading=subbibnumbered,
%     title={Confer\^{e}ncias/Eventos},
%     type=inproceedings,
% ]
%}}}
%{{{ Iniciar sem Números
\backmatter
%}}}
\end{document}
%}}}
