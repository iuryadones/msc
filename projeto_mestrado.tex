\documentclass[12pt,a4paper,oneside]{book}
\usepackage[T1]{fontenc}
\usepackage[active]{srcltx}
\usepackage[all]{xy}
\usepackage[autostyle]{csquotes}
\usepackage[brazil]{babel}
\usepackage[brazil]{varioref}
\usepackage[hidelinks]{hyperref}
\usepackage[latin1]{inputenc}
\usepackage[rflt]{floatflt}
\usepackage{alltt}
\usepackage{amsfonts,amsthm,amsopn,amssymb,latexsym}
\usepackage{amsmath}
\usepackage{appendix}
\usepackage{blindtext}
\usepackage{caption}
\usepackage{colortbl}
\usepackage{dcolumn}
\usepackage{epic}
\usepackage{epsfig,longtable}
\usepackage{epstopdf}
\usepackage{graphicx}
\usepackage{indentfirst}
\usepackage{makeidx}
\usepackage{microtype}
\usepackage{multirow}
\usepackage{pdfpages}
\usepackage{pifont}
\usepackage{ragged2e}
\usepackage{setspace}
\usepackage{subfig}
\usepackage{textcomp}
\usepackage{url}
\usepackage{wrapfig}
\usepackage{xcolor}
\usepackage{xr}

\usepackage[
    bottom=2cm,
    left=3cm,
    right=2cm,
    top=3cm,
]{geometry}

\usepackage[
    backend=bibtex,
    defernumbers=true,
    encoding=latin1,
    style=abnt,
]{biblatex}
\addbibresource{referencias/papers.bib}


\begin{document}
\begin{titlepage}
\begin{figure}[t]
\centering
    \includegraphics[width=2.5cm]{figuras/ufrpe}\\
	\label{fig:pdsmodel}
\end{figure}


\begin{center}
\vspace{1.0truecm}
\large{\textbf{UNIVERSIDADE FEDERAL RURAL DE PERNAMBUCO}}\\
\large{PR\'{O}-REITORIA DE PESQUISA E P\'{O}S-GRADUA\C{C}\~{A}O}\\
\large{PPG EM BIOMETRIA E ESTAT\'{I}STICA APLICADA}
\end{center}


\begin{center}
\vspace{5truecm}
\large{{T\'{I}TULO DO PLANO DE DISSERTA\C{C}\~{A}O}}
\end{center}

\vspace{5.0truecm}
\flushright{\large{ALUNO: IURY ADONES XAVIER DOS SANTOS}}
\flushright{\large{ORIENTADO: ADENILTON}}
\flushright{\large{CO-ORIENTADOR: PERICLES}}

\begin{center}
\vspace{3truecm}
\normal{\textsl{RECIFE - PE} - FEVEREIRO/2018}
\end{center}

\end{titlepage}
\pagebreak

\thispagestyle{empty}

\frontmatter

\tableofcontents
\pagebreak

\addcontentsline{toc}{chapter}{Indentifica\c{c}\~{a}o do plano}
\chapter*{Indentifica\c{c}\~{a}o do plano}
\renewcommand{\labelenumi}{\alph{enumi}}
\begin{enumerate}
    \item - Discente:
    \item - Bolsista: Sim ( )  N\~{a}o ( ) Ag\^{e}ncia:
    \item - N\'{i}vel:
    \item - Entrada (ano/semestre):
    \item - Orientador\(a\):
    \item - Co-orientadores:
    \item - Projeto de pesquisa do(a) orientador(a) ao qual est\'{a} vinculado o plano (projeto guarda-chuva):
    \item - \'{A}rea de concentra\c{c}\~{a}o:
    \item - Linha de pesquisa:
\end{enumerate}

\textcolor{red}{\\
* Favor apagar todas as partes em vermelho do plano de trabalho.\\
* O plano deve ter 6~8 p\'{a}ginas para mestrado e 8~10 p\'{a}ginas para doutorado, sendo contadas as folhas a partir da introdu\c{c}\~{a}o.\\
* Espa\c{c}amento 1,5 entre linhas. Margens superior e esquerda 3,0cm, inferior e direta 2,0cm.\\
* A partir da introdu\c{c}\~{a}o, imprimir em frente e verso.}
\pagebreak

\mainmatter

\chapter{Introdu\c{c}\~{a}o}
\textcolor{red}{(Deve constar obrigatoriamente revis\~{a}o de literatura, vincula\c{c}\~{a}o com a linha de pesquisa escolhida e o problema da pesquisa. Obs: exemplo da refer\^{e}ncia \autocite{einstein}, \cite[10]{einstein}, \citeauthor{einstein} e \textcite{einstein})}
\pagebreak

\chapter{Objetivos}
\textcolor{red}{(deve conter verbos sempre no infinitivo)}
\section{Geral}
\section{Espec\'{i}fico}
\pagebreak

\chapter{Metodologia}
\textcolor{red}{(Material e m\'{e}todos, descrever como ser\'{a} o processo de coleta e an\'{a}lise de dados, bem como teorias que embasam a an\'{a}lise)}
\pagebreak

\chapter{Or\c{c}amento}
\textcolor{red}{(obrigatoriamente preencher financiadores, caso haja bolsa, ajuda financeira, etc.)}
\pagebreak

\chapter{Resultados Esperados}
\begin{center}
    \begin{tabular}{|l|c|}
        \hline
        \multicolumn{1}{|c|}{Categoria} & \multicolumn{1}{|c|}{N\'{u}mero esperado} \\ \hline
        Artigo cient\'{i}fico &  \\ \hline
        Participa\c{c}\~{a}o em congresso & \\ \hline
        Patente ou equivalente & \\ \hline
        Livro ou cap\'{i}tulo de livro & \\ \hline
        Disserta\c{c}\~{a}o de Mestrado & \\ \hline
        Tese de Doutorado & \\ \hline
        Participa\c{c}\~{a}o em Editais & \\ \hline
        Outras (especificar) & \\ \hline
    \end{tabular}
\end{center}
\pagebreak

\chapter{Cronograma do discente}
\textcolor{red}{(discentes de mestrado – preencher at\'{e} o quarto semestre)}
\begin{center}
    \begin{tabular}{|l|c|c|c|c|c|c|c|c|c|}
        \hline
        \multicolumn{1}{|c|}{\multirow{2}{*}{Atividades}} & \multicolumn{8}{c|}{Semestres} \\ \cline{2-9}
         & 1$º$ & 2$º$ & 3$º$ & 4$º$ & 5$º$ & 6$º$ & 7$º$ & 8$º$ \\ \hline
        Cr\'{e}ditos de disciplinas & & & & & & & & \\ \hline
        Cr\'{e}ditos de orienta\c{c}\~ao & & & & & & & & \\ \hline
        Pesquisa Bibliogr\'{a}fica & & & & & & & & \\ \hline
        Obten\c{c}\~ao de dados & & & & & & & & \\ \hline
        An\'{a}lise de dados & & & & & & & & \\ \hline
        Profici\^{e}ncias (ingl\^{e}s/espanhol) & & & & & & & & \\ \hline
        Exame de qualifica\c{c}\~ao* & & & & & & & & \\ \hline
        Revis\~{a}o e reda\c{c}\~ao da tese & & & & & & & & \\ \hline
        Defesa da tese & & & & & & & & \\ \hline
    \end{tabular}
\end{center}
* S\'{o} para doutorado
\pagebreak

\medskip
\printbibliography[heading=bibintoc,title={Refer\^{e}ncias},type=article]
\textcolor{red}{(De acordo com a ABNT)}

\backmatter
\end{document}
